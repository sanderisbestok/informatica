%%%%%%%%%%%%%%%%%%%%%%%%%%%%%%
% LATEX-TEMPLATE GENERIEK
% Voor readme en meest recente versie, zie
% https://gitlab-fnwi.uva.nl/informatica/LaTeX-template.git
%%%%%%%%%%%%%%%%%%%%%%%%%%%%%%

%-------------------------------------------------------------------------------
%	PACKAGES EN DOCUMENT CONFIGURATIE
%-------------------------------------------------------------------------------

\documentclass{uva-inf-article}
\usepackage[dutch]{babel}
\usepackage{todonotes}

%-------------------------------------------------------------------------------
%	GEGEVENS VOOR IN DE TITEL, HEADER EN FOOTER
%-------------------------------------------------------------------------------

% Vul de naam van de opdracht in.
\assignment{Klassieke Cryptografie}
% Vul het soort opdracht in.
\assignmenttype{Samenvatting}
% Vul de titel van de eindopdracht in.
\title{Opdrachten}

% Vul de volledige namen van alle auteurs in.
\authors{Sander Hansen}
% Vul de corresponderende UvAnetID's in.
\uvanetids{10995080}

% Vul altijd de naam in van diegene die het nakijkt, tutor of docent.
\tutor{}
% Vul indien nodig de naam van de begeleider in.
\mentor{}
% Vul eventueel ook de naam van de docent of vakcoordinator toe.
\docent{}
% Vul hier de naam van de PAV-groep  in.
\group{}
% Vul de naam van de cursus in.
\course{}
% Te vinden op onder andere Datanose.
\courseid{}

% Dit is de datum die op het document komt te staan. Standaard is dat vandaag.
\date{\today}

%-------------------------------------------------------------------------------
%	VOORPAGINA EN EVENTUEEL INHOUDSOPGAVE EN ABSTRACT
%-------------------------------------------------------------------------------

\begin{document}
\maketitle
%\tableofcontents
%\begin{abstract}
%\lipsum[13]
%\end{abstract}

%-------------------------------------------------------------------------------
%	INHOUD
%-------------------------------------------------------------------------------

\section{Week 1}
\subsection{Exercises 1b}
\subsubsection{Problem 1: Caesar cipher}
(a)\\ 
Hello mister \\
7 4 11 11 14 12 8 18 19 4 17 \\
10 7 14 14 17 15 11 21 22 7 20\\
khoor plvwhu

\subsubsection{Problem 2: Alphabet creation}
a\\
A B C D E F G H I J K L M N O P Q R S T U V W X Y Z\\
A L P H B E T C R I O N D F G J K M Q S U V W X Y Z\\

\subsubsection{Problem 3: Decimation}
Legacy = A B C D E F = 11 22 33 44 55 $->$ 11 22 7 18 3 = lwhsd\\
Modern = A B C D E F = 0 11 22 33 44 $->$ 0 11 22 7 18 = alwhs\\

\subsubsection{Problem 4: Extended Euclidean Algorithm}
Voor het Extended Euclidean Algoritme maak je een tabel van vijf kolommen. De
eerste kolom is de index, de tweede de deelsom, de derde de remainder, de vierde
$s_{i}$ en de laatste $t_{i}$.\\\\
Bij index 0 en 1 wordt de quotient leeggelaten, zijn de remainders de getallen
waarmee gerekend wordt. $s_{0} = 1, t_{0} = 0, s_{1} = 0 en t_{1} = 1$.\\\\
Vervolgens is $s_{i} = s_{i-2} - s_{i-1} * quotient_{i}$ en $t_{i} = t_{i-2} - 
t_{i-1} * quotient_{i}$.

\begin{table}
    \centering
    \begin{tabular}{lllll}
    Index & Quotient & Remainder & s\_i                 & t\_i                  \\
    0     &          & 144       & 1                    & 0                     \\
    1     &          & 55        & 0                    & 1                     \\
    2     & 144:55=2 & 34        & 1-2*0=1              & 0-1*2=-2              \\
    3     & 55:34=1  & 21        & 0-1*1=-1             & 1--2*1=3              \\
    4     & 34:21=1  & 13        & 1--1*1=2             & -2-3*1=-5             \\
    5     & 21:13=1  & 8         & -1-2*1=-3            & 3--5*1=8              \\
    6     & 13:8=1   & 5         & 2--3*1=5             & -5-8*1=-13            \\
    7     & 8:5=1    & 3         & -3-5*1=-8            & 8--13*1=21            \\
    8     & 5:3=1    & 2         & 5--8*1=13            & -13-21*1=34           \\
    9     & 3:2=1    & 1         & -8-13*1=-\textbf{21} & 21--34*1=\textbf{55}  \\
    10    & 2:1=2    & 0         &                      &                      
    \end{tabular}
\end{table}

Als hier de inverse van een modulo van moet worden berekend is de laatste kolom
overbodig. Bij bijvoorbeeld $15 mod(26)$ schrijf je alleen $s_{i}$ uit.

\subsubsection{Problem 5: Playfair cipher}
Bij de playfair cipher maak je een 5x5 matrix met een codewoord. Je maakt
bigrammen van letters. Diagonalen worden vervangen door tegenovergestelde
diagonalen. Op een rij kies je voor de letter direct rechts. In een kolom voor
de letter eronder.\\
Sander wordt dus QC OC BP

\subsubsection{Problem 6: Hill cipher}
(a)

\todo{Inverse van matrix kennen?}

\section{Week 2}
\subsection{Exercises 2a}
\subsubsection{Problem 1: A simple substitution}
\section{Week 3}
\section{Week 4}
\section{Week 5}
\section{Week 6}

\end{document}
