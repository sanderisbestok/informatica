%%%%%%%%%%%%%%%%%%%%%%%%%%%%%%
% LATEX-TEMPLATE TECHNISCH RAPPORT
%-------------------------------------------------------------------------------
% Voor informatie over het technisch rapport, zie
% http://practicumav.nl/onderzoeken/rapport.html
% Voor readme en meest recente versie van het template, zie
% https://gitlab-fnwi.uva.nl/informatica/LaTeX-template.git
%%%%%%%%%%%%%%%%%%%%%%%%%%%%%%

%-------------------------------------------------------------------------------
%	PACKAGES EN DOCUMENT CONFIGURATIE
%-------------------------------------------------------------------------------

\documentclass{uva-inf-article}
\usepackage[dutch]{babel}

% Relevant voor refereren vanaf blok 5
%\usepackage[style=authoryear-comp]{biblatex}
%\addbibresource{bib}

%-------------------------------------------------------------------------------
%	GEGEVENS VOOR IN DE TITEL
%-------------------------------------------------------------------------------

% Vul de naam van de opdracht in.
\assignment{ }
% Vul het soort opdracht in.
\assignmenttype{Rapport}
% Vul de titel van de eindopdracht in.
\title{Individueel verslag}

% Vul de volledige namen van alle auteurs in.
\authors{Kevin Kuurman}
% Vul de corresponderende UvAnetID's in.
\uvanetids{11011017}

% Vul altijd de naam in van diegene die het nakijkt, tutor of docent.
\tutor{Anthony 'Toto' van Inge}
% Vul eventueel ook de naam van de docent of vakcoordinator toe.
\docent{}
% Vul hier de naam van de PAV-groep  in.
\group{A2}
% Vul de naam van de cursus in.
\course{Multimedia}
% Te vinden op onder andere Datanose.
\courseid{}

% Dit is de datum die op het document komt te staan. Standaard is dat vandaag.
\date{\today}

%-------------------------------------------------------------------------------
%	VOORPAGINA
%-------------------------------------------------------------------------------

\begin{document}
\maketitle

%-------------------------------------------------------------------------------
%	INHOUDSOPGAVE EN ABSTRACT
%-------------------------------------------------------------------------------

% Niet doen bij korte verslagen en rapporten
%\tableofcontents
%\begin{abstract}
%\lipsum[13]
%\end{abstract}

%-------------------------------------------------------------------------------
%	INTRODUCTIE
%-------------------------------------------------------------------------------

\section{Persoonlijke bijdrage}
Mijn verantwoordelijkheid was het maken van het implementeren van het spel. Het
gedeelte wat hier het meest uitdagend aan was, was het zorgen dat de twee toestellen
het spel tegelijkertijd goed weergeven.
\\Het idee was eerst om beide toestellen net zo lang te laten flitsen tot ze elkaar
herkende, dit bleek echter vrij omslachtig en in de praktijk werkte het net zo goed
wanneer we de host simpelweg te laten wachten totdat de client flitst, en dan beide
games te latne starten.
\\Om ervoor te zorgen dat de game up to date blijft versturen we elke keer een flits
wanneer er op het scherm gedrukt wordt. Dit doen we wel maar maximaal een keer in
de twee frames om te voorkomen dat twee flitsen als \'{e}\'{e}n gezien worden.
\\Om de game smooth te laten verlopen hebben we de framerate verlaagd, op deze
manier worden flitsen ook beter gedetecteerd.
\\\\Uiteindelijk ben ik ook samen met Lars aan de slag gegaan om te zorgen dat de
flitsen werden herkent. Hij liep hier bij vast en door een aantal grote aanpassingen
te maken lukte dit wel. Eerst had ik een eigen algoritme geschreven om binnen
RenderScript te kijken of het gedetecteerde licht een cirkel was. Uiteindelijk hebben
we ervoor gekozen om dit met een functie van OpenCV te doen omdat het eigen bedachte
algoritme niet werkte.


%-------------------------------------------------------------------------------
%	BIJLAGEN EN EINDE
%-------------------------------------------------------------------------------

%\section{Bijlage A}
%\section{Bijlage B}
%\section{Bijlage C}
\end{document}
